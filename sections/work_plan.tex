% Work plan: Show how you plan to organise your work, identifying immediate deliverables and dates.
\section{Work Plan}\label{sec:work_plan}

My strategy for completing the formalism outlined is composed of four milestones, with concrete and well-defined deliverables:
\begin{enumerate}
    \item Complete the formalism\\
    This will see a full formalism developed, complete with definitions.
    \item Design a test case\\
    The product of this stage will be a scenario where the formalism can be fairly tested, so as to answer the research questions as laid out in \cref{RQ}.
    \item Test the formalism\\
    This will see intelligent agents developed and tested in the example scenario. It will produce experimental data, and may result in minor modifications to the formalism so as to properly simulate responsibility.
    \item Write-up\\
    With experimental data collected and evaluated, the final report will be written up.
\end{enumerate}

\subsection{Completing the Formalism}
While some detail regarding the formalism and its structure has already been surmised, the full extent of the formalism is incomplete.\par

One aspect of the formalism which has not been developed is the mathematical reasoning which turns the semantic descriptions of the formalism into one which is fully algorithmic. This will require some deep mathematical reasoning. It will also need to draw heavily on philosophical literature, so as to create a formalism which is socially and philosophically consistent. Other work yet to be completed involves the final definition of some components, which are a necessary part of the formalism but are unlikely to have a structural impact in the way that philosophical work might.\par

The formalism should be as close to the final product of the research as possible, so extra time is allotted to complete the formalism properly. This involves finalisation of the details, and checking against literature of other disciplines. It is intended that this be completed by mid-January.\par

\subsection{Designing the Use Case}
The use case required to test the formalism is particularly important, as the somewhat semantic nature of the research may make data collection and analysis difficult. It is believed though that a test case will arise naturally from both the related trust formalisms and their experimental technique, combined with the final formalism lending itself nicely to certain problems.\par

The test case should exhibit properties of the models specifically which answer the research questions proposed (\cref{RQ}). An AI strategy should also be devised during this stage. A nuance of this segment of work is that the experiment being developed must ultimately produce some data to analyse. Therefore, metrics which are appropriate for the analysis of an artificial agent's behaviour --- and how responsibly that agent is behaving --- should be produced.\par

As this segment of the work draws heavily on definitions from the formalism and previous formalisms of human traits, it is not expected to be a significant amount of work relative to others. It is expected to be completed around the beginning of February.\par

\subsection{Testing the Formalism}
Once the test case has been identified and necessary aspects of the experiment are established, the example scenario will be developed using agent modelling tools appropriate for the task at hand. The agents developed must be designed to make use of the responsibility formalism. This can be shown, for example, by observing that agents disregard completely actions they are not responsible for. They also should identify and ignore other irresponsible agents. They should become more adept at this over time, in accordance with the research questions.\par

A concern in identifying time required for this block of work is that the formalism may prove infeasible computationally once implementation is attempted, requiring further refinement. To account for this, the analysis of the data produced should end this block of work around the middle of March.\par

\subsection{Write-up}
Once experimental data has been collected, verified, and analysed, the final block of work is to finish the write-up of the project. This segment will include writeup of analysis and production of visualisations, as well as incorporating any writing produced as the project progresses. It is expected that, should the rest of the work be properly completed at this point, that the report be finished around the beginning of April.\par

\subsection{Review of time allocation}
The time allocated to the various components of the proposed work is given in \cref{table:time-allocation}.

\begin{table}\label{table:time-allocation}
\begin{center}
\begin{tabular}{r|l}
    \emph{Section} & \emph{Intended completion time}\\
    Finalise Formalism & Mid-January\\
    Design Use Case & Beginning of February\\
    Run Tests and Collect Data & Early/Mid-March\\
    Write-up Completed & End March\\
\end{tabular}\par
    \caption{Time allocated to various parts of the work}
\end{center}
\end{table}

Note that the completion of the work ends three weeks before the project is due. To avoid complications and rushed research, the work has been planned with a three-week margin of time should any part overrun. This is due to the change that revisions of the formalism may be required, and iterating the work may be time consuming and difficult to account for. In addition, providing a small margin of time ensures minimal rushing of the project and appropriate time management indicates more quality overall work.\par

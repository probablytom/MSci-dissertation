\documentclass{article}
\usepackage{todonotes}
\usepackage{palatino}

\begin{document}

\section{Memorandum of Understanding}

A list of the work we've done so far, so as not to lost track of it all.

\subsection{Research Questions}
\begin{enumerate}
    \item How can a computational formalism of responsibility direct the decisions made by an intelligent agent?
    \item How can an intelligent agent assume the consequences of actions it makes, the decisions other agents make, and its general environment, so as to direct its interpretation of responsibility?
\end{enumerate}\par



\bigskip

\section{Basic Definitions and Terminology}

We have some basic definitions and terminology to keep track of:

\begin{enumerate}
\item An \emph{obligation} is a set of satisfaction constraints which describe something which needs to be done.\newline
  For example: tidying a room might be constrained by the satisfaction criteria: \begin{verbatim}{'room-tidiness': '>=
      0.9', 'mess-created-elsewhere': '<0.3'}\end{verbatim}

\item A \emph{responsibility} is an assigned and accepted obligation. Information relative to the assignment is bundled
  with the obligation to turn it into a responsibility. This includes:
         \begin{itemize}
             \item The assigning authority
             \item A score of importance from the authority's perspective.
         \end{itemize}

         Note that the score of importance is only determined when the responsibility is assigned: it might be
         imperative for one actor to fulfil an obligation, but not another. An example of this might be writing an
         essay. It might be imperative that one student write the essay, but optional for another.

         \item An actor which assigns an obligations is an \emph{authority}.
         \item An actor which receives responsibilities is a \emph{delegee}.
         \item When a delegee is assigned a responsibility, they may accept or reject that responsibility. A boolean
           indicating acceptance is returned to the authority.
         \item When a delegee accepts an obligation, the resulting responsibility's score is interpreted into a
           delegee-specific score via an interpretation function (for subjectivity).\newline This interpretation
           function can be a learning algorithm such as reinforcement learning, and the agent directs its interpretation
           of responsibility this way.
\end{enumerate}

\bigskip

\section{Functions to be aware of}

\subsection{Calculating other agents' responsibility}

Calculating how responsible another agent is can be done by, for every constraint which affects a resource, sum the
authority-perspective importance score multiplied by the binary form of the result of the satisfaction
constraint. Divide by the number of constraints analysed.\par

Form a vector of this calculation for each resource analysed; we end up with a vector of the responsibility of an actor
from all specific perspectives.\par

\subsubsection{Calculating an agent's own responsibility performance}

Calculating how responsible you were in past actions.

Perform this calculation by analysing your own consequential responsibilities.\par

\subsection{Calculating which agent to delegate to}

We don't necessarily want to just maximise the responsibility of the delegee, because a very generally responsible agent
may be very irresponsible for specific resources. \par

We make a matrix, where one dimension is resources, and the other actions that an agent can take. We make this matrix
for each agent. Each cell of the matrix is a representation of the affect of the action on the resource indicated by
that cell's position.\par

Then, for each agent's matrix, we analyse what actions the agent can take which discharge the given obligation
appropriately. For all agents which can discharge the obligation successfully, their specific responsibilities relative
to the resources the obligation concerns are combined\todo{How are these combined? Just summed?}. We select the delegee
by selecting the maximum specific responsibility from this pool.\par

\subsection{Influencing an interpretation of responsibility}

\ldots{}at the moment, I think the only way of doing this that's been considered is an interpretation function; nothing concrete.\par

\begin{itemize}
\item Capabilities (what is the match between the constraints of the obligation and the things I can influence in
  the environment)
  \begin{itemize}
  \item This could be calculated from each constraint, but then an extension could be proximate influence (knowing that
    doing something has an indirect effect).
  \end{itemize}
\item Some sort of global responsibility variable for each agent (so generally pre-disposed is the agent to acting responsibly?).

\item What is the match between your responsibility norms and the specifics of the responsibility (and give each norm an importance).

\end{itemize}
\end{document}

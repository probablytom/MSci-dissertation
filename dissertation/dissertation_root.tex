\documentclass{mprop}

% alternative font if you prefer
\usepackage{palatino}

% for alternative page numbering use the following package
% and see documentation for commands
%\usepackage{fancyheadings}
\usepackage{csquotes}


% other potentially useful packages
%\uspackage{amssymb,amsmath}
%\usepackage{url}
%\usepackage{fancyvrb}
%\usepackage[final]{pdfpages}

% Packages for image layouts
\usepackage{graphicx}
\usepackage{float}

\newcommand{\safealpha}{\(\alpha\)}

%% CUSTOM for todos taken from http://tex.stackexchange.com/questions/9796/how-to-add-todo-notes#9797
\usepackage[colorinlistoftodos,prependcaption,textsize=tiny,disable]{todonotes}
\newcommand{\timnote}[1]{\todo[linecolor=blue,backgroundcolor=blue!25,bordercolor=blue]{#1}}

\usepackage{enumitem}

\usepackage{cleveref}
\crefname{chapter}{\S}{\S\S}
\crefname{section}{\S}{\S\S}
\setcounter{secnumdepth}{3}
\crefname{table}{table}{tables}
\Crefname{table}{Table}{Tables}
\crefname{figure}{figure}{figures}
\Crefname{figure}{Figure}{Figures}
\crefname{appendix}{appendix}{appendices}
\Crefname{appendix}{Appendix}{Appendices}

\usepackage{float}

% \usepackage[round]{natbib}
\usepackage[style=authoryear,natbib=true,backend=bibtex]{biblatex}
\bibliography{biblio}


\begin{document}

\title{Investigating Computational Responsibility}
\author{William Wallis}
\matricnum{2025138}

\maketitle

\begin{abstract}
    Currently, models are produced for responsibility modelling which have their roots in logic. These models, while sophisticated, suffer from a lack of pragmatism: for guiding agent behaviour in sociotechnical simulations, logical models are not always ideal. In the similar field of trust modelling, algorithmic models which emulate social behaviour produce useful results while being easier to understand, implement, and reason about. In this paper\todo{paper? report? project?}, a proof-of-concept responsibility modelling platform adopting the algorithmic formalism style employed by trust modelling is produced, and its utility evaluated.
\end{abstract}

\section{Introduction}
A growing area of research lies in the formalism of human traits into computational representations. These algorithms make computers more human-like; for that reason, they are referred to here as ``Anthropomorphic Algorithms''\todo{improve the introduction of the Anthropomorphic Algorithms term}. A similar term, ``human-like computing'', has also risen in popularity lately. Human-like computing does not strictly focus on the implementation of formalisms of human traits, however, which is the area of interest for this report.\par

This implementation interest, realised in the study of anthropomorphic algorithms, presents an interesting sociotechnical problem. They present an opportunity to alter the behaviour of actors in a sociotechnical system, and to do so in a way that is easy to reason about. This alternation of behaviour is done by the algorithmic implementation of a \emph{formalism}. Formalisms present a concrete definition --- by process, mathematical definition or semantic description --- which can be used to construct an anthropomorphic algorithm. These formalisms tend to attempt to model in one of two ways:

\begin{enumerate}
    \item Modelling the trait as a useful metaphor\\
    These models tend to be inaccurate with regards the social science surrounding the trait that they model. However, they make a trade-off between this accuracy and the model's utility. For example, the notion of trust as a metaphor for a type of behaviour might be useful in information security research, but what matters in the formalisms implemented for this research is the formalism's utility in information security --- \emph{not} whether the formalism accurately represents human trust.
    \item Modelling social science directly\\
    These models attempt to accurately model the traits they concern. This can be useful for fields such as sociotechnical modelling, as well as social sciences research. There are also interesting applications for these models in interaction study: making interfaces interact with users in a human-like way, and representing the states of these traits to the users, are valuable research areas which are more applicable to these type-2 formalisms than to type-1 formalisms.
\end{enumerate}

In reality, most formalisms and their implementations lie somewhere on the spectrum that these two types define.

\section{Statement of Problem}
Computational formalisms of human traits are a growing field of research, with applications in lots of different areas. A problem with these anthropomorphic algorithms is that there is limited breadth to the scope of existing research in the field (as is demonstrated during the background survey in \cref{sec:related_work}). The metaphor of the human trait in these algorithms remains largely unexplored.\par

Breadth in the application of the metaphor is important, however. The importance stems from the utility in the human metaphor when designing systems:

\begin{itemize}
  \item Human-Computer Interaction can make use of behavioural metaphors to relay complicated internal states to a user. Storer et al.~\cite{storer_mobile_behaviour_poster} demonstrated methods by which a mobile device might dissuade certain user actions by expressing its ``discomfort'' or lack of ``trust'' in its interaction design.
  \item Information Security can make use of behavioural metaphors in order to increase difficulty of access when negative system states are encountered. A system might allow access on a graded scale, dependant on internal states of trust, comfort, and confidence.
  \item Theoretical advancements in smart city technology\cite{wallis_talk_about_x_talk} might increase a city's resilience by integrating notions of responsibility into public services and the environment on a community scale.\par
\end{itemize}

While similar results can often be achieved using regular techniques, the human metaphor allows for a better communication between a human user and complicated system states. All of the above examples center around this notion; however, the applications extend beyond Human-Computer Interaction research.\par

The lack of application of the human metaphor is a complicated mosaic of related factors. For example, research into anthropomorphic algorithms holds particular challenges, as a result of its strongly interdisciplinary nature: it requires a research team to understand the nuances of sciences as well as social sciences, and sometimes even humanities. Not only does the research team require the ability to understand these nuances, but the research must take into account their different natures. This often causes divergence in the philosophies of sociotechnical research. Some researchers view sociotechnical systems from the perspective of largely human-based systems with abstract, social behaviour. Others see sociotechnical systems as a combination of dynamic, mathematical processes which produce more technical emergent phenomena. This hints at a third complicating factor, (the second being a lack of convergence in research focus): a lack of a consistent modelling paradigm. Some research focuses largely on actor interaction-style modelling techniques~\cite{baxter2011socio}, while others rely on purely graphical modelling~\cite{ObashiMethodology}, or on mathematical modelling techniques~\cite{vespignani2012modelling}. \par

These issues together pose an issue for research in anthropomorphic algorithms: a formalism of a human-like trait is only useful to certain researchers, for certain types of models, with certain sociotechnical philosophies. Their lack of broad application is therefore unsurprising; these factors compound to produce yet another, which is that the breadth of traits formalised and researched is very small. The largest degree of research is easily conducted in the field of Trust; other traits, such as Comfort, have recently been attempted also~\cite{marsh2011defining}.\par

Recently, some interest has been shown in research pertaining to modelling and formalising \emph{responsibility}. Logical models of social trust exist which could be turned into a proof-of-concept formalism of responsibility with only a small addition: adding an obliging term in a similar way to the Deontic Logic's system for obliging\cite{deontic-logic}, allowing an agent to effectively delegate a task to another which is deemed responsible in discharging responsibilities --- the agent selected being known to be trusted already, by C\&F's already established work. The scheduling of tasks based on trust is a simple extension of existing models, or an application of existing models.\par

A model of responsibility might be more than simple task allocation, however. Some logical models of responsibility attempt to model ethically responsible decisions~\cite{berreby2015modelling}. Deontic logic's obliging term was in itself an attempt to create a logic which was suitable for the calculation of whether an agent was obliged to ensure certain goals, or perform certain tasks. Another angle might be to perceive a model of responsibility as something which might allow a responsible sociotechnical agent to choose responsibilities to discharge, rather than blindly executing tasks they are provided with in a trust-oriented model which simply delegates tasks.\par

The latter has a number of potential applications. One such possibility would be to implement agent awareness of remote task execution via RPC. Should an agent on a network be given a procedure to execute which is perceived through a responsibility formalism to be unusual, the procedure may not be executed, or may be rejected upon receipt by the responsible agent. Similar applications have proven effective in trust literature, particularly the Eigentrust algorithm~\cite{eigentrust}~, where information security is enhanced by inferring agent trustworthiness.\par

In a realistic sociotechnical system, an agent's behaviour is often informed via a feedback loop. It is important, therefore, to allow an agent to learn better ``responsible'' behaviour over time, through an analysis of its sociotechnical environment and other factors. In acknowledging that anthropic traits are most useful when they account for both introspection --- so as to direct an agent's own behaviour based on the formalised trait --- and extrospection --- so as to judge and learn from that trait in other agents.\par

To achieve these goals, two research questions were formulated:

\begin{enumerate}\label{RQ}
    \item How can a computational formalism of responsibility direct the decisions made by an intelligent agent?
    \item How can an intelligent agent assume the consequences of actions it makes, the decisions other agents make, and its general environment, so as to direct its interpretation of responsibility?
\end{enumerate}\par

Answering these questions requires the construction of a proof-of-concept formalism of responsibility which is suitable for directing agent behaviour in an algorithmic manner, as opposed to existing logical methods.

\section{Related Work}\label{sec:related_work}
A broad range of literature must be reviewed to properly understand the research at hand, due to the problem's broad nature. Particularly, this paper\todo{paper? report? project?} will focus on three areas: popular algorithmic trust models; broader sociotechnical systems research; and relevant philosophical literature.\par

\subsection{Trust Modelling}
Ordinarily, when constructing a new anthropomorphic algorithm, one would draw on literature regarding other anthropomorphic algorithms which formalise the trait being modelled. However, no such formalism exists for responsibility modelling; therefore, insights from early trust modelling may provide useful information on how to proceed, given that responsibility formalism now is in a similar stage to early trust formalism. Trust is particularly appropriate as a comparative trait to responsibility, as the two traits have many similarities (as investigated earlier).\par

\subsubsection{Marsh}
The seminal anthropomorphic algorithm for trust is found in Marsh's 1994 formalism\cite{Marsh1994FormalisingConcept}. In this paper, a formalism is described which has a number of useful qualities: it is modelled on a per-agent basis as opposed to calculating trust across a group of agents, has foundations in social sciences, and is largely algebraic, staying clear of modelling via logic.

Of particular interest is Marsh's separation of three different degrees of granularity in trust judgement. Marsh identifies that human agents have a basic degree or weighting which applies to their trust --- it would not be uncommon to assert that ``person X is very \emph{trusting}''. This is, however, distinctly different to an assertion that ``person X is trusting, but \emph{doesn't trust person Y}'' --- that is, a person's basic level of trust is distinctly different from a person's more directed degree of trust toward another agent in particular. Marsh calls this ``General Trust'', differentiating it from the earlier ``Basic Trust''. A final distinction, ``Specific Trust'', identifies an agent's degree of trust in another with regards a specific task (which could be considered ``person X's degree of trust in person Y \emph{in doing task \(\alpha\)}).\par

These separations are useful in a few key ways. One is that it identifies some of the key parameters in the model of trust: at the very least, Marsh's formalism of trust requires an agent, two agents, or two agents and a task or action in order to calculate a degree of trust. This is useful when generating a model of responsibility, as trust and responsibility share some common features: like trust, responsibility usually concerns two agents, and a task that one agent is responsible for. Unlike trust, responsibility modelling contains hierarchies: one agent might be considered the authoritative figure, which \emph{delegates} a task to another. For the purposes of this paper\todo{paper? report? dissertation? project?}~, the latter agent will be referred to as a ``delegee''. The similarities in the parameters of the formalism affirm the notion that responsibility and trust are similar in concept, and also serve as a starting point for imagining what an algorithmic formalism of responsibility might be like.\par

Another useful insight from these separations would be that, when considering responsibility, judging how responsible an agent is can be done from the same three vantage points. Basic responsibility would colloquially be ``benefit of the doubt'', general responsibility would be how responsible an agent in all modelled capacities, and specific responsibility would be another agent's calculated responsibility with regards some task, resource, or other modelled subject of responsibility). Not only does this also affirm trust and responsibility's similarities, but it helps in providing further structure to the new formalism.\par

A final important property of Marsh's formalism is its graded nature. This model does not produce a boolean indicator of whether to trust or not; rather, it opts to calculate a number between 0 and 1 of the \emph{degree} of trust one agent ought to have in another. This feature allows for as granular a model of trust as any given application requires, and allows for more nuanced comparison between agents.\par

\todo{Write more of this? Maybe less on the separations, and more on its social science background or lack of logical foundation?}

\subsubsection{Castelfranchi \& Falcone}
Another useful formalism of trust to consider comes from Castelfranchi \& Falcone~\cite{CastelfranchiSocialApproach} (often referred to as ``C\&F theory''). While this formalism has logical foundations, its large popularity makes it an interesting comparison to Marsh's formalism.\par

In contrast to Marsh's formalism, C\&F is graded only in more complex forms. At its root, C\&F is a logical formalism built on boolean calculations of goal and belief state satisfaction. However, C\&F also segregate different aspects of trust in their formalism: different elements of the basic formalism represent competence, disposition, and dependence. They define competence as one agent's belief and will that another agent can successfully achieve a goal by a certain action, disposition as the belief and will that the other agent is willing to perform an action to achieve a goal, and dependence as the effective delegation of a task for the purpose of achieving a goal (expressed via logical statements).\par

Curiously, the C\&F model of trust contains all of the same parameters as Marsh's, plus a fourth: the goal to be achieved by an action. One may take the position that a goal is also important in modelling responsibility --- but it is not immediately apparent that it is necessary to include it. Therefore, C\&F uncovers the need when building a formalism of a trait to make choices as to the formalism's philosophy regarding the trait.\par

In particular, two philosophical differences between Marsh's model and that of C\&F are immediately apparent: where Marsh separates his calculation of trust into a basic/general/specific granularity, C\&F separate trust's different constituent parts, each of which must be satisfied for trust to be present. Another philosophical difference between the two is the assertion as to whether the goal of an action factors into one's trust. These philosophical decisions can make concrete, important differences to a formalism's constitution --- which is plain to see when considering that even the parameters of the model differ, something fundamental to the formalism's definition of trust. Therefore, these choices as to the approach to responsibility are important to note in the formalism produced during this report\todo{report? paper? Dissertation?}.\par

\subsubsection{Eigentrust}
Eigentrust\cite{eigentrust} is a formalism which takes a notably different approach to Marsh and C\&F's formalisms: rather than attempting to model human trust accurately, it models trust as a metaphor for the truly desired behaviour.\par

Calculations of trust in Eigentrust have their foundations in eBay's model for reputation: star ratings based on a summation of satisfaction scores. In particular: \[s(i,j) = sat(i,j) - unsat(i,j)\]~\ldots{}where \(sat\) is the number of satisfactory interactions between two agents and \(unsat\) the unsatisfactory ones, represents a ``local trust value'' that an agent \(i\) has in another agent \(j\). Through some linear algebra, these local trust values are accumulated and gradually turned into a global score of responsibility, accounting for all agents' opinions of each other. This global score could also be considered to be an agent's \emph{reputation} --- in this way, Eigentrust generalises trust as a certain application of a reputation formalism, and bootstraps its own formalism on another trait.\par

While Eigentrust has proven particularly effective in its intended domain, it is a formalism designed expressly for the purposes of network and information security. Eigentrust achieves this by a number of philosophical differences to Marsh and C\&F's respective formalisms:

\begin{itemize}
    \item Eigentrust operates in a distributed way. Unlike the local scoring system employed by Marsh and C\&F's formalisms, Eigentrust has all agents report their local trust scores, so as to create a distributed ledger of more general trust scores.
    \item Eigentrust does not model trust directly, nor does it claim to model it accurately --- unlike C\&F's formalism, which is strictly intended as a model of human behaviour, and Marsh's, which simulates it, Eigentrust uses ``trust'' as a description of an agent's behaviour.
    \item Eigentrust is not concerned with the cause of satisfactory or unsatisfactory interactions; it focuses entirely on accumulated positive/negative scores. In this way, Eigentrust somewhat models Marsh's ``general trust'', but makes no assertions as to trust's composition or how to reason about it.
\end{itemize}

In these respects, Eigentrust represents an interesting alternative end of the spectrum between anthropomorphic algorithms which treat their trait as a metaphor, or as a social behaviour to accurately simulate. While Eigentrust does not represent a very useful foundation for our responsibility formalism, it does highlight two things:

\begin{enumerate}
    \item The responsibility formalism required to test the research questions should be more similar to Marsh and C\&F's formalisms than Eigentrust: there is no specific use case to design for, so designing with a trait as a metaphor would not be appropriate.
    \item Should it be necessary, a trait's formalism could be bootstrapped using a pre-existing formalism of another trait. Whether this is a suitable way to answer the research questions above is harder to address; the possibility should therefore be considered.
\end{enumerate}

\subsubsection{FIRE}
FIRE\cite{huynh2004fire} is another trust modelling system which provides a focus on being able to judge trust using information from many different sources. For example, it treats direct experience  information in a different way to information collected by third parties. It is also a model which considers multiple different traits: it incorporates reputation information into its judgement of trust. In part, FIRE is able to do this because it segregates different information sources into different measurements, which are tabulated into a score after their measurement.\par

FIRE's inclusion of information from multiple sources paints other trust formalisms in a slightly different light. However, this feature is not necessary for all trust scenarios: the decision to trust (or not trust) is made with different amounts of information for agents in different simulations. One can imagine a two-agent simulation, where information about each agent's interactions would be assessed by exactly one source --- the other agent, which judges the former's reliability. It is plain to see that FIRE is designed to be a formalism treating traits as a metaphor, similarly to Eigentrust.\par

This philosophical decision makes FIRE an unlikely candidate as a foundation of a responsibility formalism, as its specific application area --- sociotechnical models with multiple types of information to consider --- is more complex than is necessary for a proof-of-concept responsibility formalism, and would be more complicated than the research questions posed require. However, the philosophical choice it raises regarding different types of information, and the nature of the information which is being reviewed when calculating the responsibility score, is an important one.\par

% TODO: should this be 'sociotechnical systems', 'Ian Sommerville', or 'Responsibility Modelling'? What's a good title? Is there anything other than Sommerville to go in here?
\subsection{Sociotechnical Systems}
While these anthropomorphic algorithms are useful to consider in isolation, their application within the realm of sociotechnical systems is important to their design. Moreover, the nature of responsibilities is touched within the broader sociotechnical systems area of responsibility modelling --- research on the delegation and discharge of responsibilities, and how to reason about them.\par

\subsubsection{Ian Sommerville}
\todo{What more can we say about Sommerville's contributions? At the moment we only cover consequential and causal responsibilities.}
Ian Sommerville was a prolific writer in the field of sociotechnical systems, who was responsible for much of the current literature on responsibility modelling\cite{sommerville_graphical_responsibility,sommerville_dependable_systems_chap_8,sommerville_dependable_systems_chapter_9}.\todo{What other references can go here?} Sommerville's responsibility modelling systems often happened to be graphical\cite{sommerville_graphical_responsibility}, a paradigm for computational responsibility which may prove hard to convert to an anthropomorphic algorithm. This is because graphical representations of system states do not naturally present themselves as a numerically analysable format for information --- rather, graphical presentations are useful for exposing sociotechnical system state. This feature of sociotechnical modelling, and particularly responsibility modelling, is useful for risk and impact analysis\cite{ObashiMethodology}.\par

Sommerville's writing, however, presents a wealth of interesting insights which may be useful in understanding the context of the anthropomorphic algorithm, and understanding some possible choices as to the formalism's philosophy.\par

In particular, Sommerville notes an interesting distinction as to two different sorts of responsibilities: ``consequential'' and ``causal'' responsibilities. They can be thought of as the difference between a responsibility for a current state --- the result of a previously discharged responsibility --- and responsibility for producing a future state --- a change to a current state that an agent is responsible for bringing about in the future.\par

Sommerville's separation clarifies a potential avenue by which one might create a computational responsibility formalism. Namely, an agent's degree of responsibility might be associated with the state changes they have brought about in the past, and perhaps a prediction of the chance that an agent would discharge a causal responsibility successfully at some future time. This approach, verified by its appearance in existing sociotechnical systems literature, may be appropriate for creating a computational responsibility formalism, though its foundations exist in a less useful graphical representation approach.\par

\subsubsection{Tim Storer and Russell Lock}
While describing a graphical responsibility modelling format for the InDeED project --- notably lead by Ian Sommerville --- Storer and Lock provide some useful foundations for other models of responsibility in a technological report on modelling responsibility\cite{storer2008modelling}.\par

Most interestingly, Storer and Lock provide a useful definition of a responsibility:

\begin{quotation}
    A duty, held by some agent, to achieve, maintain or avoid some given state, subject to conformance with organisational, social and cultural norms.\cite{storer2008modelling}\par
    
    \ldots{}\par
    
    Responsibilities are the duties to be discharged by agents as described\ldots{}
\end{quotation}

Numerous things in this definition are useful. Similarly to the work done by Mash and C\&F, the definition provides an insight into some possible fundamental parameters of a responsibility:

\begin{description}
    \item [A duty: ] responsibilities can be considered as some action an agent is \emph{obliged} to conduct.
    \item [achieve\ldots{}some given state: ] the obligation which a duty is defined by can be described as a change of \emph{state}. Storer and Lock note that this state change can have different modes: it can be achieved, but also maintained or avoided. It is worth noting that this very general description of responsibility indicates that the graphical formalism designed for the InDeED project was somewhat socially accurate.
    \item [comformance with\ldots{}norms: ] agents can be delegated responsibilities in most useful formalisms of responsibility --- Storer and Lock note that these responsibilities must not conflict with norms that an agent holds.
\end{description}

Unlike Marsh or C\&F's more mathematical definitions of trust, this semantic definition lends itself nicely to conversion to some responsibility model; partly because it already summarises the responsibility trait, but also because it describes an intuitive social definition of responsibility while remaining very simple. Therefore, a responsibility formalism which somehow extended this model, while retaining the power of the anthropomorphic algorithms described earlier would provide a very useful starting point for the desired responsibility formalism.\par

Another useful definition from Storer and Lock's report presents itself when discussing the structure of a responsibility in their model:

\begin{quotation}
    Responsibilities may be composed of other responsibilities.
\end{quotation}

This composibility is an interesting property, which may also imply that some responsibilities can be broken down into a form of sub-responsibility. This property would be useful to remember when modelling responsibilities in the desired formalism. It also implies that there may be a sort of ``atomic'' responsibility, that more complicated responsibilities are composed of.

\subsection{Philosophical Literature}  % The nature of responsibility, and the validity of modelling anthropic traits in a machine.
As responsibility is a social trait, and previous literature has indicated that it may be most appropriate to model it as such, input from literature in the humanities may shed light on how to consider and approach responsibility as a trait accurately. As time allowed for this report was somewhat limited, a more complicated sociologically/psychologically accurate model was not attempted. The decision was made to instead create a proof-of-concept model which encapsulated much of the existing sociotechnical systems literature, and verifying it using philosophical literature on the subject.

\subsubsection{Thomas Scanlon}  % A philosophical perspective which converges on Sommerville's ideas in Dependable Systems
Thomas Scanlon --- a philosopher who specialises in Analytic Philosophy --- writes on responsibility in the essay ``Justice, Responsibility, and the Demands of Equality'', defining terms which bear striking resemblance to Sommerville's ``Consequential'' and ``Causal'' responsibilities.\par

One term --- ``attributive'' responsibility --- is defined by Scanlon as being:

\begin{quotation}
    What a person sees as a reason for acting, thinking, or feeling a certain way\cite{scanlon2006justice}
\end{quotation}

This might be generalised as being the responsibility for future and current action, as attributive responsibilities would clearly be the influencing responsibility of a future action. If future actions discharge the responsibilities an agent posesses at a given time, then Scanlon's ``attributive'' responsibilities in effect mirror Sommerville's ``causal'' responsibilities.\par

Scanlon goes on to define a second term, ``substantive'' responsibility, which can be described loosely as a responsibility for an action fixed in the past. Given the sociotechnical perspective of responsibibility being a change of sociotechnical state, Scanlon's ``substantive'' responsibilities also seem to mirror a term of Sommerville's, namely his ``consequential'' responsibilities.\par

Given the convergence of this perspective on responsibility --- that is, the separation of past state change and anticipated future state change --- Scanlon's writing seems to affirm Sommerville's own construction of types of responsibility --- though clarity on how this distinction is useful is found in work from another philosopher, P.F. Strawson.\todo{I feel like this segue is pretty poor! How can I improve this?}

\subsubsection{P.F. Strawson}  % The nature of responsibility
P.F. Strawson is a philosopher with work on philosophy of language and of mind --- one particular essay, ``On Freedom and Resentment''\cite{strawson}, presents an argument that determinism shouldn't affect how we percieve human factors such as responsibility or trust.\par

Strawson argues that these concepts are fundamentally relational, rather than having any inherant definition. The conclusion of this is that trust, responsibility, and other human traits have subjective properties and cannot be calculated outside of the perspective of an agent. This hints at a choice as to the desired formalism's philosophy: unlike a formalism such as Eigentrust, where components of trust scores are calculated independently of any one agent's perspective, the required responsibility scoring system must utilise a subjective perspective for \emph{all} calculations. Strawson's argument applies to sociotechnical systems, as it does not apply to any specific trait or type of person; therefore, both social and technological actors in a sociotechnical system might be the targets of a trait designed with Strawson's work in mind.\par

Strawson also produces a fairly rigorous analysis of how ordinarily variable human traits can be formalised:

\begin{quotation}
    Indignation, disapprobation, like resentment, tend to inhibit or at least to limit our goodwill towards the object of these attitudes, tend to promote an at least partial and temporary withdrawal of goodwill; they do so in proportion as they are strong; and their strength is in general proportioned to what is felt to be the magnitude of the injury and to the degree to which the agents will be identified with, or indifferent to, it.\cite{strawson}
\end{quotation}

Strawson here identifies a similar property of human traits to Marsh: when an agent acts, the magnitude of the judgement of a trait with respect to the act is proportional to the magnitude of its effect on an overall judgement of the trait. That is to say, if an action has a given importance, the magnitude of that importance should carry through to judgements that the action impacts. This, combined with the subjectivity implied by the earlier point made by Strawson, implies that importance might be interpreted subjectively.\par

The introduction of a parameter of importance associated with a responsibility score allows the magnitude of importance of an act to be quantified and tracked in consequential or substantive responsibilities, adhering to Strawson's second point. Should some mechanism for interpretation be provided by the formalism, a subjective factor is introduced, satisfying Strawson's first point. This also hints at how an agent might judge the degree of responsibleness of itself or another agent, helping to answer the second research question; Strawson's argument here is therefore useful in the construction of the desired computational responsibility formalism.\par

\subsubsection{Sloman}  % The space of artificial minds, and a brief note on the philosophical value of this formalism (optional!)



\section{A Formalism of Responsibility}

\subsection{Design Decisions and Philosophy}

% Decision to be socially accurate
%  - There's no specific application area, so we should focus on modelling responsibility as a general trait, rather than solving a specific problem. 
%  - The research questions are best answered by a non-application-oriented formalism, because they're describing aspects of a responsibility formalism, not a responsibility formalism in a certain setting!
%  - We're not trying to solve any issue perceived in other anthropomorphic algorithms, like FIRE does.

% Decision to be decentralised --- no need to communicate judgement scores, agents calculate all on their own.

% Decision to judge responsibility based on the success/failure of an action to meet an action's requirements (in answer to the question FIRE raises, on what sort of information the formalism considers)
%  - Based on the literature describing consequential/causal responsibility.
%  - We calculate causal responsibilities based on the known consequential responsibilities.

% Decision not to care about act, but to describe outcome (effectively the goal)

% Decision to make the formalism subjective
%  - Different agents in the real world have different ``opinions'' about responsibility
%  - If the model is to be socially accurate to a degree, we should include a mechanism for different agents to see different responsibilities differently.

\subsubsection{Constraints}  % Constraints are the "atomic responsibilities" defined earlier. It's the fudamental unit of state change that Storer describes earlier.

\subsubsection{Obligations}  % Sets of constraints. An obligation is a description of a responsibility's effect, but doesn't concern itself with metadata like the discharging agent.

\subsubsection{Responsibilities}  % An assigned obligation, with an authority and a delegee. Explain that importances are also introduced here, as the same obligation may be important to different delegees in different ways. Properly define the term "discharge"

\subsubsection{Agents}  % Briefly describe Theatre model for agents and their workflows. Define authorities and delegees.

\subsubsection{Directing agent decisions}  % Explain the interpretation function, and how it is used to make the entire formalism subject as per Strawson's writings.
% Note here that, given that the agent can use the responsibility formalism's interpretation function to guide its behaviour in a responsible way (and in a potentially better way than simple task scheduling), the responsibility formalism guides their behaviour pretty effectively and according to the social trait.

\subsubsection{Judging agent responsibility}  % Judging an agent's repsonsibleness based on their consequential responsibilities. Differences between basic, general, and specific responsibilities.
% Also, note here that judging agent responsibilities allows an agent to decide whether they need to change their behaviour, as they can sample the judgements over time and find whether they're more/less responsibly judged when they make certain decisions -- satisfying RQ2.


\section{Evaluating the Formalism}

\subsection{RQ 1: Acting on Responsibilities}

\subsection{RQ 2: Judgement of Responsibility}



\section{Future Work}

\subsection{Advancing the formalism}


\section{Discussion}


\bibliographystyle{abbrv}
\bibliography{biblio}

\end{document}
:W